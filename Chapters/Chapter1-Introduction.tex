\chapter{Introduction}

\label{Chapter1}

\section{Pretext}

AUVIC, is a team of students whose objective is to design, build, and compete an
\gls{AUV} in the yearly RoboSub competition in San Diego, California.
The team provides students with opportunity for hands on experience in submarine
design, and work on complex autonomous systems.
This kind of experience opens the clubs members up for many opportunities for
Co-ops and post graduation careers. 

The team has evolved in its membership durastically over the last year, as many
long standing members graduate, and take on Co-op jobs, and new less experienced
members take the reigns.
The club as a whole has a small number of students in relation to previous years
which limits what we can accomplish.
The club is composed of three different teams: mechanical, electrical, and
software.
The report will be dealing entirely with the software side of the AUV.



\section{Report Contents}

This report documents the migration from an existing control system.
The report starts off with explaining and understanding the problem domain from the
perspective of the control system, which will help to inform the reader about
what the operational constraints are of the system, and the task which it is
trying to accomplish.
Software re-engineering occupies a majority of the report and servers for the
discussion, and meat of the report, this section covers the different processes
that will inform the software redesign.
Those processes being: reverse engineering, restructuring, and forward engineering.

After the design and implementation the report concludes with reflection on the
software system produced from the redesign.
This is done in two parts, a conclusion which talks about the results of the
system featured at competition, which outlines the strengths and weaknesses of
the new system abstractly.
The report then concludes the reflection with a detailed analysis of what failed
and succeeded and suggestions of how to fix the failure that was outlined.
