\chapter{Problem Domain}

\label{Chapter2}

\section{Operational Environment}

The development for the AUV's high-level software system is developed in C++ and
python. These languages are used as they work with our development environment.
Our development consists of two primary libraries, \gls{ROS} and \gls{opencv},
running on Ubuntu 16.04. ROS is a amalgamation of different libraries useful for
the development of robotic systems. We primarily use ROS for inter-process
communication management, and hardware abstraction. OpenCV is a software package
for the processing of images and video, used to give the AUV understanding from visual inputs.

The existing system consists of several different packages, two of which are
impacted by the development of a new control system; those being "ai" and "nav."
The ai package handled the state machine functionality of the AUV, also handling
the detection systems as well. While the nav package handles direction messages
from the ai packages' control system segment.

Existing systems.
Inter-connectivity through ROS communication structures (that diagram).
Programming languages, and technologies.

Hardware.
Hydrophones, hydrophone boards, power board, IMU, Jetson TX2.
Diagram of hardware connections.

The ai package manages the information need to inform the control system about
the external state. Currently this is done with camera input and depth input
from the power board. This was intended to be expanded to support other devices,
such as hydrophones and an \gls{IMU}.
