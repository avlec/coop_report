\chapter{Problem Domain}

\label{Chapter2}

\section{Operational Environment}

To start understanding how the system works we first need to understand the
environment in which the system operates.
The system runs in Ubuntu 16.04 on an NVIDIA Jetson TX2 Development kit.
This operating system is required to support the use of the \gls{ROS} Kinetic
software, its use will be explained more below.
Development of the system is written in C++ and python.
Most of the code written on the AUV is C++ this is done primarily to increase
the performance of the concurrent systems, and this languages' type system
ensures code correctness.
Python, being used to a lesser extent, is used for image processing and the web
interface for managing the system.
Using Python for image processing seems sub-optimal when the system uses C++
because of its performance benefits.
However, the image processing interfaces for Python are more simple than C++,
and since the Python code doesn't need to be compiled (C++ needs to be compiled)
the Python code can be tested and adjusted significantly faster than the C++
code.
These languages are used as they work with our development environment which
consists of two primary libraries, \gls{ROS} and \gls{opencv}.
ROS acts as the backbone for the AUV as it is used to launch all the different
required sub-processes, and manage the inter-process communication between them.
ROS also aids with hardware abstraction, but that isn't seen much in this
report as the report focuses on the higher level components of the system.
OpenCV is used to handle retrieving still images from the cameras and processing
of those images for image recognition, pattern and feature matching, and cascade
filtering to detect various objects for the RoboSub competition.

The existing system consists of several different packages, two of which are
impacted by the development of a new control system; those being "ai" and "nav."
The ai package handled the state machine functionality of the AUV, also handling
the detection systems as well. While the nav package handles direction messages
from the ai packages' control system segment.

Existing systems.
Inter-connectivity through ROS communication structures (that diagram).
Programming languages, and technologies.

Hardware.
Hydrophones, hydrophone boards, power board, IMU, Jetson TX2.
Diagram of hardware connections.

The ai package manages the information need to inform the control system about
the external state.
Currently this is done with camera input and depth input
from the power board.
This was intended to be expanded to support other devices,
such as hydrophones and a \gls{IMU}.
The other devices used currently, camera and power board, have proven to give
enough control from use in previous competitions.
However, the addition of the IMU would allow for more accurate measurement of
the orientation of the submarine; which would allow for a more accurate heading.
Furthermore, the addition of hydrophone functionality would allow for the AUV to
speed up execution of some tasks at competition.
This is because the competition course provides two underwater pingers (sound
emitting sources) that can be used to navigate quickly to sections of the
course.
As currently the AUV must search for those sections of the course with the
cameras.
